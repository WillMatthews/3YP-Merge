%\documentclass[11pt,oneside]{article}
%\usepackage{style-3yp}
%\lfoot{Lai Chong}
\let\subsection\paragraph
%\begin{document}

\section{Conclusion}
The main objective of this project is to design an ammonia-based ESS to power the island of Maui using wind energy harvested from a windfarm. The plant designed consists of six major components in order to achieve the goal: hydrogen formation by electrolysis of water, nitrogen separation from air by cryogenic distillation, ammonia synthesis by Haber-Bosch process with iron-based catalyst, ammonia gas turbine, solid oxide fuel cell network and power control system. With a planned lifetime of 25 servicing years, the plant is designed to be sustainable and fully self-sufficient. All raw materials are naturally obtainable, and no harmful byproducts are produced or released in waste, purge or exhaust streams. 
\subsection{Meeting the Demand}
The power demand of Maui is to be met by the application of a hybrid power generation system: SOFCs and gas turbine. Gas turbine (with max power of 270MW) has a much shorter rise time than SOFCs (with max power of 230MW), and can be used as a temporary power supply as SOFCs ramp up. Hence as SOFCs are warming up, gas turbine can smooth out the transition for power generation and accommodate for short-term power spikes; SOFCs can then takeover and provide continuous power generation to meet baseline demand. Both units have built-in failsafe capacity and can be ran in conjunction in the event of high power demand, and is therefore sufficient to meet the fluctuating demand profile of 100-190 MW as required.
\subsection{Material Balance}
Around 90,000 tpy of ammonia is produced from 74,000 tpy of nitrogen from cryogenic distillation and 16,000 tpd of hydrogen from electrolysis of water, which is stored and later decomposed to release energy. To allow for possibility of demand fluctuation, power demand and ammonia supply are not perfectly matched; a marginal amount of excess ammonia and hydrogen is produced which can be flared to achieve a zero overall material balance.
\subsection{Sustainability}
To assess the plant's sustainability, the plant is judged upon two main criteria: environmental impact and societal impact. \\
With reference to each design component's environmental sustainability assessment: \\
\begin{enumerate}
    \item Main inputs are air and seawater, which are abundant and naturally obtainable.
    \item Minimal amount of $CO_2$ produced from separation of air does not contribute towards the increase in net $CO_2$ level in the atmosphere.
    \item Other waste products of the plant are high-purity nitrogen, oxygen and water, and are all non-toxic in nature.
    \item The only direct source of Greenhouse gas emission is from the gas turbine, which will only be ran 5\% of the time.
\end{enumerate}
On the societal side, the construction and operation will create jobs and boost the local economy of Maui, with full access to clean energy available upon the operation of the plant. The plant will allow Maui to reach its renewable energy targe of 100\% by 2045.\\
To summarise, the sustainability analysis of the plant has shown that it is highly sustainable, and that it has a minimal environmental impact whilst achieving an overall positive societal impact.
\subsection{Safety and Risk}
Due to the operating conditions and chemical risks involved in a plant of this magnitude, safety and risk aspects of individual plant components are carefully considered, researched and presented in their respective "Safety and Risk" sections. HAZOP studies have been conducted to components where applicable, highlighting likely modes of failure as well as devising counter-measures to reduce the risks.
\subsection{Financial Analysis}
The cost estimations for the plant are performed using the 'Total Purchasing Cost Estimate'. Total purchase cost of major plant components and various aspects of the plant's construction and cost of operation are estimated to be \$6680 million, with the largest shares of component purchase cost from the windfarm and SOFCs at around \$1000 million individually. Levelised cost of electricity is \$0.17 per kWh of electricity produced, which is relatively competitive compared to the average sale price of \$0.28 per kWh in Hawaii, despite being much higher than US national average of \$0.12 per kWh. The financial analysis of the plant has shown that it is highly feasible for the location chosen, and that it is relativity competitive compared to alternative renewable energy sources.
\subsection{Future Research}
 Analysis of the proposed design showed that the project is highly feasible in areas of meeting power demand, sustainability and profit turning. In reality, EES of this scale is highly complex and will require more in-depth and rigorous investigation, e.g. refined controller design, more accurate thermodynamic and box emission assessment. This report offers an idealistic design draft of ESS, such that a more intricate design can be pursued in the future.
%\end{document}