\documentclass[11pt,oneside]{article}
\usepackage{style-3yp}
\lfoot{Lai Chong}
\let\subsubsubsection\paragraph
\begin{document}

\section{Conclusion}
\noindent The main objective of this project is to design an ammonia-based ESS to power the island of Maui using wind energy harvested from a windfarm.    The plant designed consists of six major components in order to achieve the goal: hydrogen formation by electrolysis of water, nitrogen separation from air by cryogenic distillation, ammonia synthesis by Haber-Bosch process with iron-based catalyst, ammonia gas turbine, solid oxide fuel cell network and power control system. Designed to be sustainable and fully self-sufficient, all raw materials are naturally obtainable. No harmful byproducts are produced or released in waste, purge or exhaust streams. 
\subsection{Meeting the Demand}
\noindent The power demand of Maui is to be met by the application of a hybrid power generation system: SOFCs and gas turbine. Gas turbine (with max power of 270MW) has a much shorter rise time than SOFCs (with max power of 230MW), and can be used as a temporary power supply as SOFCs ramp up. Hence as SOFCs are warming up, gas turbine can smooth out the transition for power generation and accommodate for short-term power spikes; SOFCs can then takeover and provide continuous power generation to meet baseline demand. Both units have built-in failsafe capacity and can be ran in conjunction in the event of high power demand, and is therefore sufficient to meet the fluctuating demand profile of 100-190 MW as required.
\subsection{Material Balance}
\noindent 246.8 tpd of ammonia is produced from 202.8 tpd of nitrogen from cryogenic distillation and 44.0 tpd of oxygen from electrolysis of water, which is stored and later decomposed to release energy. To allow for possible of demand fluctuation, power demand and ammonia supply are not perfectly matched; a marginal amount of excess ammonia is produced which can be flared to achieve a zero overall material balance.
\subsection{Financial Analysis}
The cost estimations for the plant are performed using the 'Total Purchasing Cost Estimate'. Total purchase cost of major plant components and various aspects of the plant's construction and cost of operation are estimated to be \$6680 million. Levelised cost of electricity is \$0.17 per kWh of electricity produced, which is relatively competitive compared to the average sale price of \$0.28 per kWh in Hawaii, despite being much higher than US national average of \$0.12 per kWh. The 
\subsection{Sustainability}
\subsection{Future Research}
\noindent Analysis of the proposed design showed that the project is highly feasible in areas of meeting power demand, sustainability and profit turning. In reality, EES of this scale is highly complex and will require more in-depth and rigorous investigation, e.g. refined controller design, more accurate thermodynamic and box emission assessment. This report offers an idealistic design draft of ESS, such that a more intricate design can be pursued in the future.
\end{document}