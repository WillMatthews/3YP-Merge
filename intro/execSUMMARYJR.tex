%\documentclass[11pt, a4paper]{article}

%\title{Plant Safety}
%\author{James Rhodes}
%\date{\today}
%\usepackage{style-3yp}


%\usepackage{graphicx}
%\graphicspath{ {graphics/} } 
%\linespread{1.5}




%\begin{document}

%\maketitle
%\tableofcontents
%{\renewcommand{\arraystretch}{1.0}}
%\doublespacing
\section{Executive Summary}

This report provides an analysis and evaluation of the current and prospective  sustainability, profitability and design requirements of an ammonia-based energy storage system for Maui, HI, USA. Methods of analysis include model simulation and validation, as well as plant costings and supply/demand matching. Other analysis include plant safety, sustainability as well as potential environmental impacts.

The report covers a range of current and emerging technologies and assesses their respective strengths and weaknesses. Currently, there is significant opportunity for the deployment of an energy storage system capable of meeting the island's 1.31 TWh annual demand whilst maintaining a profit margin of up to 64\% at current energy prices. Total purchase cost is currently estimated to be \$6680 million.


The report finds than in order to best meet the initial criteria, the preferred design stages are the use of electrolysis to produce hydrogen and cryogenic air separation to produce nitrogen, using excess energy harvested from a wind farm. These can then be fed into an ammonia synthesis reactor producing liquid ammonia for storage. This in turn can be used to power a solid oxide fuel cell system and gas turbine to generate electricity on demand. The minimum lifetime of the plant is taken to be 25 years.

The report also investigates the limitations of the scope of the project when conducting analysis. Some of the limitations include:
\begin{itemize}
    \item The dependence on energy prices remaining high in Maui relative to the US average.
    \item The full environmental impact of the capital purchases have not been analysed.
    \item The decommissioning costs of the plant have not been considered.
    \item The impact of natural disasters beyond tropical storms have been excluded from the report.

\end{itemize}



%\bibliography{./SafetyJR/v3bib}
%\bibliographystyle{abbrv}
%\end{document}