%\documentclass[11pt, oneside]{article}  

%\usepackage{style-3yp} %this is the .sty file
%\usepackage{pdflscape}
%\lfoot{Enzia Schnyder} %your name in the footer

%table
%\usepackage{array}
%\newcolumntype{L}{>{\centering\arraybackslash}m{2.5cm}}
%\newcolumntype{J}{>{\centering\arraybackslash}m{2.8cm}}

%\begin{document}
\section{Sustainability}

A principal design target for the storage plant is that it must be sustainable. The plant impacts assessed in this section aim to judge if the plant:
\begin{itemize}
\item \textbf{Has a minimal environmental impact} - this includes low resource consumption, waste, pollutant emission and greenhouse gas emissions.
\item \textbf{Has an overall positive social impact} - the social value of the plant must exceed the cost
\end{itemize}
Issues associated with each design component are covered in their respective sections. This section will summarise these issues and include additional issues associated with the entire plant.

The plant will allow the whole of Maui to switch to a constant source of renewable energy, which currently make up only 24.8\% of Maui's energy \cite{website:mauielectric}. This would reduce lifecycle GHG emissions by up to 700 tonnes CO2e/GWh \cite{GHG}. The only direct greenhouse gas emissions from the ESS plant are small amounts of $NO_x$ from the gas turbine, which will only run 5\% of the time and indirect emissions from transport and construction of the plant. The main waste products of the plant are pure oxygen, nitrogen and water which are not polluting gases. The main inputs are air and seawater, which are abundant in the plant location. Therefore, the plant represents a large improvement to Maui's current environmental impact.

The construction and operation of the plant will require significant manpower. This creates jobs, boosting the local economy of Maui. Residents will also benefit from gaining access to clean energy. Although there will be minor visual and noise pollution caused by the plant, there will still be an overall positive societal impact.

%\printbibliography[heading=subbibliography]
\bibliography{./turbine/Sustainability}
\bibliographystyle{unsrt}
%\end{document}